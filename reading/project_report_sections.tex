% -----------------------------------------------------------------------------
% Project Report �C Key Sections
% This .tex file provides standalone sections that can be input-included into
% the main report. The structure mirrors the reference report while replacing
% content with the present team's information (Group-16).
% -----------------------------------------------------------------------------

% -----------------------------------------------------------------------------
% 1. Product Requirement Breakdown
% -----------------------------------------------------------------------------
\section{Product Requirement Breakdown}
\label{sec:prd}

\subsection{Product Definition and Expectation}
A small, lightweight remote-controlled vehicle capable of sensing four distinct
signal types (infra-red, radio-frequency, magnetic field, and ultrasonic) and
feeding back the information accurately and quickly to the operator. The design
prioritises \textbf{accuracy over speed}.

\subsection{System-Level Thinking}
\begin{itemize}
    \item \textbf{Power Supply}: Four~\SI{1.5}{\volt} batteries (VBAT) regulated down to
          \SI{5}{\volt} and \SI{3.3}{\volt} rails through LDOs.
    \item \textbf{Sensor Interface}: Digital output preferred; analogue fallback where
          unavoidable.
    \item \textbf{Remote Communication}: Wi-Fi or NRF24 modules, with a clear
          front-/back-end abstraction inside the MCU firmware.
    \item \textbf{Mechanical Considerations}: Compact footprint, balanced weight
          distribution, DC motor selection, driver choice, battery placement
          and cooling.
\end{itemize}

\subsection{Development Principles}
\begin{description}[leftmargin=1.5cm,labelsep=0.4cm]
    \item[Fast\hfill\small Be resourceful] Utilise parts already available in the
          laboratory; pre-order components to minimise lead-times.
    \item[Iteration matters] Produce a simple but working prototype early;
          iterate and improve in small, measurable steps. \emph{Stability 
          $\gg$ fancy features.}
    \item[Efficient on Budget] Read the datasheets, avoid over-engineering, and
          select suppliers that balance cost with reliability.
\end{description}

% -----------------------------------------------------------------------------
% 2. Team Setup \& Roles
% -----------------------------------------------------------------------------
\section{Team Setup \& Roles}
\label{sec:team}

\begin{table}[H]
    \centering
    \caption{Team members and primary responsibilities}
    \begin{tabular}{@{} l l p{8cm} @{}}
        \toprule
        \textbf{Name} & \textbf{GitHub/Handle} & \textbf{Key Responsibilities} \\
        \midrule
        Martin Jingchuan-Liang & \texttt{martinJL} & Acting team lead; RF~\& ultrasonic sensor design; mechanical / hardware fabrication; component procurement and budget tracking. \\
        Jonas Zheng & \texttt{Phantomkiiid} & Infra-red and Hall-effect sensor design; mechanical CAD modelling; final assembly. \\
        Minsoo Kwak & \texttt{qwertykwak} & Co-development of all four sensors; hardware debugging and bring-up; PCB routing and assembly. \\
        Clementine White & \texttt{clementinewhite} & Vehicle kinematics and movement algorithms; board�Cserver communication layer; ADC and UART sensor interfaces. \\
        Emir Alemdar & \texttt{Emiral46} & Web-based front-end development; UI/UX design; integration testing with embedded back-end. \\
        Eugene Irwin & \texttt{KRISTO} & Wi-Fi communication stack and remote control; high-rate digital sampling for sensors; co-development on wheel-base CAD. \\
        \bottomrule
    \end{tabular}
    \label{tab:team_roles}
\end{table}

% -----------------------------------------------------------------------------
% 3. Project Management
% -----------------------------------------------------------------------------
\section{Project Management}
\label{sec:management}

\subsection{Expectations and Milestones}
The project launched on \textbf{12~May}. Key delivery milestones are summarised
in Table~\ref{tab:key_dates}.

\begin{table}[H]
    \centering
    \caption{Key dates and deliverables}
    \begin{tabular}{@{} l l @{}}
        \toprule
        \textbf{Date}\hspace{2cm} & \textbf{Milestone / Deliverable} \\
        \midrule
        11�C17~May & All four sensors complete: schematic design, simulation, and prototyping tests. \\
        21~May & Improved CAD design; software front-end and back-end skeleton finished. \\
        23~May & Full vehicle assembly; system-level integration; sensor reading verified. \\
        \bottomrule
    \end{tabular}
    \label{tab:key_dates}
\end{table}

\subsection{Planning Approach}
A rolling two-week Gantt schedule (Fig.~\ref{fig:gantt}) is maintained to track
progress. Tasks are logged with clear predecessors and resource assignments.

% -------------------------------------------------
% Example of including the Gantt chart slide
% -------------------------------------------------
\begin{figure}[H]
    \centering
    % Replace the path below with the actual exported image / PDF of the Gantt.
    \includegraphics[width=0.95\linewidth]{figs/gantt_chart.png}
    \caption{Planning Gantt chart (excerpt)}
    \label{fig:gantt}
\end{figure}

\subsection{Communication and Collaboration}
The team employs a three-layered documentation and communication stack:
\begin{enumerate}
    \item \textbf{Notion} -- engineering logbook and design documentation.
    \item \textbf{GitHub} -- version control for firmware, software, and CAD; issue tracking and project boards for planning.
    \item \textbf{WhatsApp} -- daily updates, rapid troubleshooting, and ad-hoc discussions.
\end{enumerate}
All formal decisions are reflected in Notion and linked to the corresponding
GitHub issues to maintain traceability.

% -----------------------------------------------------------------------------
% End of file
% ----------------------------------------------------------------------------- 